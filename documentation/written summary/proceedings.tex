\documentclass{sigchi}

% Use this section to set the ACM copyright statement (e.g. for
% preprints).  Consult the conference website for the camera-ready
% copyright statement.

% Copyright
% \CopyrightYear{2016}
%\setcopyright{acmcopyright}
% \setcopyright{acmlicensed}
%\setcopyright{rightsretained}
%\setcopyright{usgov}
%\setcopyright{usgovmixed}
%\setcopyright{cagov}
%\setcopyright{cagovmixed}
% DOI
% \doi{}
% ISBN
% \isbn{}
%Conference
% \conferenceinfo{}{}
%Price
% \acmPrice{}

% Load basic packages
\usepackage{balance}       % to better equalize the last page
\usepackage{graphics}      % for EPS, load graphicx instead 
\usepackage[T1]{fontenc}   % for umlauts and other diaeresis
\usepackage{txfonts}
\usepackage{mathptmx}
\usepackage[pdflang={en-US},pdftex]{hyperref}
\usepackage{color}
\usepackage{booktabs}
\usepackage{textcomp}


% Some optional stuff you might like/need.
\usepackage{microtype}        % Improved Tracking and Kerning
% \usepackage[all]{hypcap}    % Fixes bug in hyperref caption linking
\usepackage{ccicons}          % Cite your images correctly!
% \usepackage[utf8]{inputenc} % for a UTF8 editor only

% If you want to use todo notes, marginpars etc. during creation of
% your draft document, you have to enable the "chi_draft" option for
% the document class. To do this, change the very first line to:
% "\documentclass[chi_draft]{sigchi}". You can then place todo notes
% by using the "\todo{...}"  command. Make sure to disable the draft
% option again before submitting your final document.
\usepackage{todonotes}

% Paper metadata (use plain text, for PDF inclusion and later
% re-using, if desired).  Use \emtpyauthor when submitting for review
% so you remain anonymous.
\def\plaintitle{Maze Gaze - A Gaze-Based Multiplayer Game}
\def\plainauthor{Kevin Müller, Marco Siweris, Lena}
\def\emptyauthor{}
\def\plainkeywords{eye-tracking; gaze-based interaction; multi-user interaction; human computer interaction;}
\def\plaingeneralterms{Documentation, Standardization}

% llt: Define a global style for URLs, rather that the default one
\makeatletter
\def\url@leostyle{%
  \@ifundefined{selectfont}{
    \def\UrlFont{\sf}
  }{
    \def\UrlFont{\small\bf\ttfamily}
  }}
\makeatother
\urlstyle{leo}

% To make various LaTeX processors do the right thing with page size.
\def\pprw{8.5in}
\def\pprh{11in}
\special{papersize=\pprw,\pprh}
\setlength{\paperwidth}{\pprw}
\setlength{\paperheight}{\pprh}
\setlength{\pdfpagewidth}{\pprw}
\setlength{\pdfpageheight}{\pprh}

% Make sure hyperref comes last of your loaded packages, to give it a
% fighting chance of not being over-written, since its job is to
% redefine many LaTeX commands.
\definecolor{linkColor}{RGB}{6,125,233}
\hypersetup{%
  pdftitle={\plaintitle},
% Use \plainauthor for final version.
%  pdfauthor={\plainauthor},
  pdfauthor={\emptyauthor},
  pdfkeywords={\plainkeywords},
  pdfdisplaydoctitle=true, % For Accessibility
  bookmarksnumbered,
  pdfstartview={FitH},
  colorlinks,
  citecolor=black,
  filecolor=black,
  linkcolor=black,
  urlcolor=linkColor,
  breaklinks=true,
  hypertexnames=false
}

% create a shortcut to typeset table headings
% \newcommand\tabhead[1]{\small\textbf{#1}}

% End of preamble. Here it comes the document.
\begin{document}

\title{\plaintitle}

\numberofauthors{3}
\author{%
  \alignauthor{Kevin M{\"u}ller\\
    \affaddr{Saarland University}\\
    \affaddr{Saarbr{\"u}cken, Germany}\\
    \email{s9kvmuel@stud.uni-saarland.de}}\\
  \alignauthor{Marco Siveris\\
    \affaddr{Saarland University}\\
    \affaddr{Saarbr{\"u}cken, Germany}\\
    \email{e-mail address}}\\
  \alignauthor{Lena Hornberger\\
    \affaddr{Saarland University}\\
    \affaddr{Saarbr{\"u}cken, Germany}\\
    \email{s8lehorn@stud.uni-saarland.de}}\\
}

\maketitle

\begin{abstract}
This paper is the result of the Seminar "Multi-User Gaze-Based Interaction" at the Saarland University. We designed and implemented an interactive application which uses gaze input. The result of this development is a gaze based multiplayer game called "Maze Gaze" in the category "Multi-User Content Adaption". In this paper we present our project idea, the requirements we made, some design and software decisions and show that a game with gaze input can make a lot of fun despite working with a eyetracker can cause a lot of issues.
\end{abstract}

% \keywords{\plainkeywords}

\section{Introduction}
1/4 page intro (Kevin)
\subsection{Motivation}
1/2 page (Kevin)
\subsection{Goal}
1/4 page (Kevin)
\subsection{Outline}
1/4 page (Kevin)\\
---\\
TOTAL: 1.25 Pages

\section{Related Work}
\subsection{Gaze Input Performance}
1/2 page (Kevin)
\subsection{Mouse Against Gaze Input}
1/2 page (Kevin)
\subsection{Immersive Game Controll Using Gaze}
The paper "Gaze-Controlled Gaming: Immersive and Difficult but not Cognitively Overloading" was published in 2014 from Kreytz et al. in Warsaw, Poland and present at the UbiCom in September 2014 in Seatlle, USA.\\
The paper is about controlling games through eye-movements.\\
In the experiment the participants should guide a character trough a maze with their eye movements. It was design within-subjects and had two fixed factors. The first was the game-control type which can be differ in 3 types: gaze-controlled with cues, gaze-controlled without cues and keyboard-controlled. The second factor was the maze complexity(easy and hard). Each participants played all three versions of the game two times, once with the easy and once with the hard maze.\\
The results important for our project: the completion time was faster with cues, but there were more saccades and there were more saccades with the complex maze, but the participants were gazing on the paths of the maze about 60\% of the time.\\
This resulted the main question: moving the character or scanning paths? For our project this means, that we are switching between two modes: gaze-controlled gaming an visual field scanning. This means we only move the character when the gaze is within a given radius. Moreover we do the game without cues that show possible directions, because this reduce the game experience. 
\subsection{Pursuit Calibration}
The paper "Pursuit Calibration: Making Gaze Calibration Less Tedious and More Flexible" was published in 2013 from Pfeuffer et al. and present at the UIST in October 2013 in St. Andrews, United Kingdom.\\
As we know, every application which works with gaze input needs a user calibration before you can interact with it. But a "normal" calibration maybe is difficult and tedious. It requires five or more calibration points. In this paper they present another method to calibrate: "Pursuit Calibration" at which the users are following moving targets to calibrate.\\
In the experiment they test how the calibration should be: They test different speeds of the moving target and the time the target moves. First the target moves with a constant speed and second with a accelerating moving target which slows down at the corner of the screen and accelerates at the straight side of the screen. The results were: The target does not have the travel across the entire screen, because the final accuracy is already reached around 66\% of its whole path and durations that lasts longer than 10 seconds are better.\\
This resulted how the calibration should be: it should introduce users to the application and be very intuitive (e.g in a stargazing application, following shooting stars). The duration of the calibration should be between 10 to 20 seconds and the size of the target should be, in our case, the width of the path, because the size of the target has an impact on the accuracy of the calibration.\\
In our project the pursuit calibration was a may have we didn't implemented.

---\\
TOTAL: 1.5 Pages

\section{Requirements and Design}
\subsection{Gaze Interaction Design Decisions}
1/2 page (Marco) (Farben, Content Adaption (private public shared areas), radius)
\subsection{Agile Development Approach}
1/4 page (Marco)
\subsection{Formal Requirements}
1/2 page (Marco)\\
---\\
TOTAL: 1.25 Pages

\section{Implementation}
1/4 page intro
\subsection{Hardware}
1/4 page (Kevin)
\subsection{Software}
2 pages
\subsubsection{Menu} Lena
\subsubsection{Maze Generator} Marco
\subsubsection{Power-Ups} Marco
\subsubsection{Smart Player Controls} Kevin
\subsubsection{Game Fairness} Kevin
\subsubsection{Pupil Integration} Kevin
\subsubsection{Calibration} Kevin
\subsubsection{Game Lobbys} Kevin
\subsubsection{Graphical Interface} Marco
---\\
TOTAL: 2.5 pages


\section{Evaluation}
\subsection{Performance}
1/4 page (Lena)
\subsection{User-Experience}
1/4 page (Lena)\\
---\\
TOTAL: 0.5 page

\section{Conclusion}
\subsection{Résume}
1/4 page (Lena)
\subsection{Lessons Learned}
1/4 page (Kevin)
\subsection{Future Work}
1/2 page (Marco)\\
---\\
TOTAL: 0.75 page

\section{TOTAL: 8 pages}

\balance{}


% REFERENCES FORMAT
% References must be the same font size as other body text.
\bibliographystyle{SIGCHI-Reference-Format}
\bibliography{sample}

\end{document}

%%% Local Variables:
%%% mode: latex
%%% TeX-master: t
%%% End:
