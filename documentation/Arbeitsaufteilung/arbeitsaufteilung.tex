\documentclass[10pt,a4paper]{article}
\usepackage[latin1]{inputenc}
\usepackage{amsmath}
\usepackage{amsfonts}
\usepackage{amssymb}
\usepackage{graphicx}
\usepackage[width=19.00cm, height=28.00cm, left=2.00cm, right=2.00cm, top=2.00cm, bottom=2.00cm]{geometry}
\begin{document}
	\section*{Kevin}
		\begin{itemize}
			\item Koordination der Lichtquellen, ob das kontinuierliche Licht des Spielers, Area Licht oder Target Licht
			\item Spieler so bearbeiten, so dass diese sich nicht gegenseitig blockieren k�nnen
			\item Algorithmus entwickelt zum fairen spawnen des Power-ups darunter befindet sich ein Algorithmus zum um den k�rzesten Pfad zwischen zwei Zellen zu berechnen
			\item Kompletter Umgang des Eye Trackers, von Verbindungsaufbau bis hin zur Steuerung (welche Orientiert ist zur Maussteuerung), setzen der Marker, etc...
			\item Einbetten der Offline und In-Game Kalibrierung
			\item Anwendung entwickelt, damit Spieler in einem laufenden Spiel hinzutreten k�nnen und ab der n�chsten Runde mitspielen k�nnen
			\item Verbesserung der Gazeposition an den R�ndern, in Zusammenarbeit mit Marco
			\item Anpassen von Text- und Schriftgr��en
		\end{itemize}
	\section*{Marco}
		\begin{itemize}
			\item Generierung eines zuf�lligen Labyrinthes, welches jedoch von jedem Startpunk jeden Punk des Labyrinthes erreichen kann
			\item Erstellen des erster Spielers und Implementierung der Steuerung des Spielers via Maus 
			\item Ausbau auf 4 Spieler, welche nun per Tastatur gesteuert werden k�nnen (Pfeiltasten, WASD, UHJK)
			\item Spiel kann gewonnen werden: Implementation daf�r, dass wenn jemand das Target gesammelt hat, sich das Startmen� wieder �ffnet und man wieder eine Runde Spielen kann. Sp�ter folgte Ausbau auf Rundenbasiertes Spiel, so dass man im Menu schon die Rundenanzahl w�hlen konnte, man nach diesen Runden jedoch wieder neue Einstellungen t�tigen konnte und wieder Spielen kann, ohne das Spiel neu zu starten.
			\item Methode zum fairen spawnen der Power-Ups, so dass die Power-ups nicht nur einmal erscheinen. Jedoch auch mit angepassten Zeitabstand. Es folgten Verbesserungen, so dass die Power-Ups nicht auf einem Target spawnen k�nnen.
			\item Content Adaption implementieren, also wenn ein Nutzer auf einen Bereich eines anderen Spielers schaut, wird dieses Licht f�r den Moment ausgeschaltet, wenn er weg schaut, erleuchtet es wieder. 
			\item Ausw�hlen und einbetten von Sound, das betrifft klick eines Button, Hintergrundmusik, so wie Soundeffekte, wenn ein Spieler ein Power-Up oder Target sammelt, Best�tigungs- oder Fehlsignal zur Kalibrierung, Start einer neuen Runde, etc... 
			\item Implementierung zum aus dem Spiel kicken eines Spielers, wenn er sich f�r l�ngere Zeit nicht bewegt, also der Nutzer den Spieler nicht steuert.
			\item Leichte Beschleunigung der Spieler implementieren
			\item Da Gazedaten an den R�ndern des Spielfeldes ungenau sind, Behebung des Problems ins Zusammenarbeit mit Kevin
			\item L�sung daf�r, dass das Target nie auf einem Powerup und immer im dunklen Bereich spawned.
			\item Hinzuf�gen von 3D-Grafiken f�r Power-Ups und Target, sowie SPX wenn ein Power-Up oder Target gesammelt wurde und Implementierung der zuf�llig spawnenden Nebelwolke �ber dem Spielfeld. 
		\end{itemize}
	\section*{Lena}
		\begin{itemize}
			\item Erstellen des Men�s vom Grundaufbau bis zum Finalen Men�
			\item Implementierung eines Pause-Buttons
			\item Hinzuf�gen eines Winner-Screens, welcher im Nachhinein durchgehen an den R�ndern angezeigt wird
			\item Ideen aus Related Work zusammenfassen und Anwendungen auf unser Spiel herausfiltern
			\item Optimierungen im Men� f�r besseres User-Interface-Design
			\item Ausf�hrliches Testen
			\item Ranking Liste hinzuf�gen (durchgehender Punktestand der Spieler, welcher sortiert ist)
			
		\end{itemize}
\end{document}